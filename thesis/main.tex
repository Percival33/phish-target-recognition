%-----------------------------------------------
%  Engineer's & Master's Thesis Template
%  Copyleft by Artur M. Brodzki & Piotr Woźniak
%  Warsaw University of Technology, 2019-2022
%-----------------------------------------------

\documentclass[
    bindingoffset=5mm,  % Binding offset
    footnoteindent=3mm, % Footnote indent
    hyphenation=true    % Hyphenation turn on/off
]{src/wut-thesis}

\graphicspath{{tex/img/}} % Katalog z obrazkami.
\addbibresource{bibliografia.bib} % Plik .bib z bibliografią
\RequirePackage{minted}
\RequirePackage[
    colorinlistoftodos,
    textsize=scriptsize
]{todonotes}
\RequirePackage{marginnote}
\let\marginpar\marginnote
\RequirePackage{placeins}

%-------------------------------------------------------------
% Wybór wydziału:
%  \facultyeiti: Wydział Elektroniki i Technik Informacyjnych
%  \facultymeil: Wydział Mechaniczny Energetyki i Lotnictwa
% --
% Rodzaj pracy: \EngineerThesis, \MasterThesis
% --
% Wybór języka: \langpol, \langeng
%-------------------------------------------------------------
\facultyeiti    % Wydział Elektroniki i Technik Informacyjnych
\EngineerThesis % Praca inżynierska
\langpol % Praca w języku polskim

\begin{document}

%------------------
% Strona tytułowa
%------------------
\instytut{Informatyki}
\kierunek{Informatyka}
\specjalnosc{Sztuczna Inteligencja}
\title{
    Rozpoznawanie celu podszycia stron phishingowych
}
% Title in English for English theses
% In English theses, you may remove this command
\engtitle{
    Unnecessarily long and complicated thesis' title \\
    difficult to read, understand and pronounce
}
% Title in Polish for English theses
% Use it only in English theses
\poltitle{
    Niepotrzebnie długi i skomplikowany tytuł pracy \\
    trudny do przeczytania, zrozumienia i wymówienia
}
\author{Marcin Jarczewski}
\album{330234}
\promotor{prof. dr hab. inż. Wojciech Mazurczyk}
\date{\the\year}
\maketitle

%-------------------------------------
% Streszczenie po polsku dla \langpol
% English abstract if \langeng is set
%-------------------------------------
\cleardoublepage % Zaczynamy od nieparzystej strony
\abstract \lipsum[1-3]
\keywords phishing, rozpoznawanie wizualne, sieci syjamskie

%----------------------------------------
% Streszczenie po angielsku dla \langpol
% Polish abstract if \langeng is set
%----------------------------------------
\clearpage
\secondabstract \kant[1-3]
\secondkeywords XXX, XXX, XXX

\pagestyle{plain}

%--------------
% Spis treści
%--------------
\cleardoublepage % Zaczynamy od nieparzystej strony
\tableofcontents

%------------
% Rozdziały
%------------
\cleardoublepage % Zaczynamy od nieparzystej strony
\pagestyle{headings}

\clearpage % Rozdziały zaczynamy od nowej strony.
\section{Wstęp}
\subsection{Wprowadzenie}
Wraz z rozwojem internetu i postępem cyfryzacji istnieje coraz więcej stron internetowych, a co za tym idzie, więcej oszustw internetowych.
Powstają strony phishingowe, które mają na celu wyłudzanie poufnych danych poprzez specjalnie przygotowane wiadomości e-mail lub SMS, 
które zawierają linki prowadzące na specjalnie przygotowane strony internetowe, łudząco podobne do tych prawdziwych. 
Zapobieganie tej formie cyberprzestępczości jest jednym z zadań zespołu CERT Polska, pełniącego obowiązki CSIRT\footnote{\textit{Computer Security Incident Response Team} - Zespół Reagowania na Incydenty Bezpieczeństwa Komputerowego}  NASK\cite{certpl}. 
Do tej pory stosowano różne sposoby zapobiegania oszustwom, w których stosowano podejścia blacklist \cite{google-blacklist}, (również CERT Polska udostępnia listę\footnote{\href{https://cert.pl/lista-ostrzezen/}{https://cert.pl/lista-ostrzezen/}} podejrzanych domen). 
Oprócz tego wykorzystywane są rozwiązania opierające się na kodzie HTML danej strony np. \cite{PhishZoo}, jednak rozwiązania te są mało efektywne, ponieważ podobne wizualnie strony mogą mieć zupełnie inny kod źródłowy. 

TODO: dodać jakieś inforamcje jak to wygląda w innych krajach np albo jakiś rys historyczny :xD: tak żeby wspomnieć na wstępie o tym że jest to współpraca z CERT Polska

\subsection{Cel pracy}
TODO: albo motywacja pracy
Motywacją tej pracy jest przybliżenie tematyki phishingu, sposobów rozpoznawania na podstawie podobieństwa wizualnego oraz przygotowanie systemu który grupuje istiejące metody, umożliwiając ich ewaluacje. System ten będzie stanowił inspiracje oraz jako podstawa do rozbudowy w przyszłości przez zespół CERT by chronić polski internet przed kampaniami phishingowymi.

W tej pracy zostanie zbadana tematyka rozpoznawania stron phishingowych na podstawie podobieństwa wizualnego. Jest to szczególnie ważne dla zespołu CERT Polska, który zajmuje się wykrywaniem, informowaniem o potencjalnych zagrożoniach oraz udaremnianiem takich ataków. Wobec powyższych zagrożeń, istotne jest dla zespołu CERT posiadanie narzędzia, które będzie w stanie skorzystać z dotychczasowych metod wykrywania phishingów by jak najdokładniej określić czy strona jest phishingowa a jeśli tak się zdarzy, to rozpoznać cel podszycia. Zatem dodatkowym zadaniem postawionym w tej pracy stworzenie systemu, który będzie realizować wyżej określone zadania oraz będzie umożliwiał dołożenie nowych metod niewielkim nakładem pracy co przekłada się na modularność systemu w zakresie wykorzystywanych metod rozpoznawania jak również w zakresie wykorzystywanych danych wejściowych.
Zbadanie tematyki rozpoznawania stron phishingowych na podstawie podobieństwa wizualnego oraz stworzenie systemu, który będzie w stanie rozpoznać cel podszycia strony phishingowej.

Celem pracy jest również wykorzystanie uczenia maszynowego, który będzie w stanie na bazie wyników z różnych metod wydać ostateczną decyzję, że coś jest lub nie jest phishingiem.

\subsection{Struktura pracy}

TODO: poprawić ten tekst, czas przyszły czy przeszły?

W rozdziale drugim zostanie przedstawiona wiedza teoretyczna, na której opierają się zastosowane rozwiązania (?) z przeglądu literatury naukowej w tym zakresie. Przegląd ten został on przeprowadzony w rozdziale trzecim. Dalej w rozdziale czwartym najpierw wymienione są wymagania wobec systemu i decyzje projektowe, które zostały podjęte aby takowe wymagania spełnić. Kolejny rozdział traktuje o implementacji systemu, w którym opisane są technologie w zrealizowanym systemie a także opis w jaki sposób zostały one zintegrowane ze sobą. W rozdziale szóstym opisane są eksperymenty, które zostały przeprowadzone w celu sprawdzenia skuteczności systemu. Zostały tam porównane każda z metod na każdym zbiorze danych z literatury naukowej oraz na zbiorze przygotowanym przez CERT. Na końcu znajduje się podsumowanie pracy, w którym zawarte są wnioski z przeprowadzonych eksperymentów oraz możliwości dalszego rozwoju systemu.

% % Akapit z cytatem
% \lipsum[1] \cite{goossens93}

% % Przykładowy obrazek
% \begin{figure}[!h]
%     % Wyrównanie obrazka, szerokość i plik
%     % Zamiast width można też użyć height, etc.
%     \centering \includegraphics[width=0.5\linewidth]{logopw.png}
%     % Podpis umieszczamy pod obrazkiem
%     % znacznik \caption służy również do wygenerowania numeru obrazka
%     \caption{Tradycyjne godło Politechniki Warszawskiej}
%     % \label pozwala odwołać się do obrazka w innych miejscach za pomocą \ref
%     % odwołanie \ref renderuje się jako numer obrazka,
%     % dlatego zawsze najpierw używaj \caption a potem \label
%     \label{fig:tradycyjne-logo-pw}
% \end{figure}

% % \ref wyrenderuje się jako 'Reference to image 1.1'
% \lipsum[2] Reference to image \ref{fig:tradycyjne-logo-pw}.

% % Lista punktowana
% % Parametr label ustawia symbol punktora
% \begin{itemize}
%     \item Item 1:
%     \begin{itemize}[label=---]
%         \item item 1.1;
%         \item item 1.2;
%     \end{itemize}
%     \item Item 2;
%     \item Item 3.
% \end{itemize}

% \lipsum[3]

% % Lista numerowana w formacie 1.a).ii
% % Tutaj również można stosować \label
% \begin{enumerate}
%     \item Item 1:
%     \begin{enumerate}
%         \item item 1.1;
%         \item item 1.2:
%         \begin{enumerate}
%             \item item 1.2.1;
%             \item item 1.2.2;
%         \end{enumerate}
%         \item item 1.3;
%     \end{enumerate}
%     \item Item 2;
%     \item Item 3.
% \end{enumerate}

% % Przypis dolny \footnote
% \lipsum[4] Lorem ipsum dolor sit amet\footnote{Lorem ipsum dolor sit amet, consectetur adipiscing elit, sed do eiusmod tempor incididunt ut labore et dolore magna aliqua. Ut enim ad minim veniam, quis nostrud exercitation ullamco laboris nisi ut aliquip ex ea commodo consequat.}, consectetur adipiscing elit.

% % Przykładowa tabela: wyśrodkowana i renderowana
% % w miejscu wstawienia: !h = !h[ere]
% % Domyślnie tabele trafiają na górę strony
% \begin{table}[!h] \centering
%     % Podpis tabeli umieszczamy od góry
%     \caption{Przykładowa tabela.}
%     \label{tab:tabela1}

%     % Tabela z trzema kolumnami:
%     % dwie wyrównanie do środka [c], a ostatnia do prawej [r]
%     % szerokość kolumn automatyczna (równa szerokości tekstu)
%     \begin{tabular}{| c | c | r |} \hline
%         Kolumna 1       & Kolumna 2 & Liczba \\ \hline\hline
%         cell1           & cell2     & 60     \\ \hline
%         cell4           & cell5     & 43     \\ \hline
%         cell7           & cell8     & 20,45  \\ \hline
%         % Komórka o szerokości dwóch kolumn, wyrównana do prawej
%         % Przypisy dolne w tabelach wstawiamy przez \tablefootnote
%         \multicolumn{2}{|r|}{Suma\tablefootnote{Table footnote.}} & 123,45 \\ \hline
%     \end{tabular}

% \end{table}

% Lorem ipsum dolor sit amet.

\clearpage % Rozdziały zaczynamy od nowej strony.
\section{Podstawy teoretyczne}
W tym rozdziale scharakteryzowano podstawy teoretyczne zagrożenia typu phishing jak i pojęć związanych z automatycznycmi systemami, które są wykorzystywane w pracy.
\todo[inline]{uzupełnić wstęp teoretyczny potrzebny do zrozumienia tematyki - co to jest phishing, jakie są metody jego rozpoznawania - pojęcia, pomysły, sieci konowucyjne, sieci syjamskie + transfer learning, deskryptory wizualne, hasze wizualne}

\noindent\makebox[\linewidth]{\rule{\paperwidth}{0.4pt}}

W tym rozdziale zostaną omówione zagadnienia związane z stronami phishingowymi, ochronie iternautów oraz inne zagadnienie teoretyczne potrzebne do zrozumienia tematu. 

istniejące sposoby rozpoznawania stron phishingowych na podstawie różnych atrybutów. 
Przytoczone rozwiązania określają czy strona internetowa jest stroną phishingową, korzystając z cech takich jak analiza kodu strony, adresu URL czy też rozpoznawanie wizualne. W dalszej części tego rozdziału zostanie omówiona tematyka zbiorów danych. 
\subsection{Phishing}
- co to jest
- przykłady
- kto jest celem

\todo[inline]{dodać jakieś inforamcje jak to wygląda w innych krajach np albo jakiś rys historyczny :xD: tak żeby wspomnieć na wstępie o tym że jest to współpraca z CERT Polska}


\subsection{CERT Polska}
- co to jest CERT Polska
- czym się zajmuje i jaki to ma związek z phishingiem
- jakie są inne instytucje zajmujące się tym problemem
- strona gdzie można zgłaszać informacje o phishingu

pełniącego obowiązki CSIRT\footnote{\textit{Computer Security Incident Response Team} - Zespół Reagowania na Incydenty Bezpieczeństwa Komputerowego}  NASK\cite{certpl}

Do tej pory stosowano różne sposoby zapobiegania oszustwom, w których stosowano podejścia blacklist \cite{google-blacklist}, (również CERT Polska udostępnia listę\footnote{\href{https://cert.pl/lista-ostrzezen/}{https://cert.pl/lista-ostrzezen/}} podejrzanych domen). 
Oprócz tego wykorzystywane są rozwiązania opierające się na kodzie HTML danej strony np. \cite{PhishZoo}, jednak rozwiązania te są mało efektywne, ponieważ podobne wizualnie strony mogą mieć zupełnie inny kod źródłowy. 

\subsection{Kod źródłowy}
Istnieje wiele rozwiązań, które próbują określić cel ataku na podstawie adresu URL i jego atrybutów. W pracy \cite{KnowYourPhish} autorzy opierają się na słowach kluczowych występujących na stronie i porównaniu z nazwą podmiotu wynikającego z adresu URL na podstawie \textit{fqdn}
\footnote{fully qualified domain name -  pełna, jednoznaczna nazwa domenowa, określająca położenie danego węzła w systemie DNS. - \href{https://pl.wikipedia.org/wiki/Fully_Qualified_Domain_Name}{https://pl.wikipedia.org/wiki/Fully\_Qualified\_Domain\_Name}}.
\begin{figure}[!h]
    \centering
    \includegraphics[width=0.5\linewidth]{FQDN-example.png}
    \caption{Struktura adresu URL z pracy \cite{KnowYourPhish}}
    \label{fig:Struktura adresu URL z pracy}
\end{figure}

gdzie:
\begin{itemize}
    \item protocol = https
    \item FQDN = www.amazon.co.uk
    \item RDN = amazon.co.uk
    \item mld = amazon
    \item FreeURL = {www, /ap/signin? encoding=UTF8}
\end{itemize}

Inne prace \cite{PhishWHO} wykorzystują słowa kluczowe i wyszukiwarkę internetową porównując najwyżej notowane wyniki przy wyszukaniu słów kluczowych na danej stronie by w ten sposób określić cel podszycia strony. Kolejnym podejściem jest analiza linków wychodzących z badanej strony i sprawdzenie czy prowadzą na nią z powrotem, budując w ten sposób graf. Na jego podstawie określany jest cel podszycia jednocześnie stwierdzając czy strona jest prawdziwa.

\begin{figure}[!h]
    \centering
    \includegraphics[width=0.8\linewidth]{Antiphishing through phishing discovery.png}
    \caption{Graf zbudowany z linków dostępnych na badanej stronie z \cite{PhishWHO}}
    \label{fig:enter-label}
\end{figure}

\subsection{Hasze wizualne i deskryptory}

W zastosowaniach kryptograficznych do jednoznacznej identyfikacji pliku wykorzystywane są funkcje skrótu takie jak: MD5, SHA-1, SHA-2. Dla wejścia o dowolnym rozmiarze uzyskujemy skrót stałego rozmiaru.
Zależy nam na minimalizacji liczby kolizji, czyli przypadków gdy otrzymano taki sam skrót dla różnych danych wejściowych.

Do wykrywania podobnych informacji by wykrywać naruszenia praw autorskich lub wykrywać nielegalne treści, wykorzystuje się technikę zwaną \textit{Locality-sensitive hash}, które mają na celu pogrupowanie elementów tak, że podobne do siebie trafiają do jednej grupy. W tym celu, przeciwnie niż do zastosowań kryptograficznych maksymalizowana jest liczba kolizji. Tak aby bardziej podobne dane wejściowe uzyskały bliższe sobie wartości skrótu, uzyskując przy tym odcisk palca - \textit{fingerprint} \cite{Zauner2010ImplementationAB}.

\todo[inline]{czym są deskryptory?}
\todo[inline]{co to są zanurzenia}
\subsection{Sieci neuronowe}
Z racji że rozwiązania opierające się wyłącznie na rozpoznawaniu stron na podstawie kolorów \cite{EMD} jest mało skuteczne, autorzy prac skupiają się na zastosowaniu sieci głębokich a dokładniej sieci konwolucyjnych.

\todo[inline]{opisać sieci syjamnskie}

Sieć konwolucyjna (CNN) jest to klasa sztucznych sieci neuronowych, najczęściej wykorzystywana w zadaniach wizji komputerowej, ze względu na efektywne przetwarzanie danych o strukturze przestrzennej. Kluczowym elementem sieci są warstwy konwolucyjne. Każda z nich zawiera zestaw filtrów, które wykonują operację konwolucji na danych wejściowych. Filtry te uczą się rozpoznawać określone cechy, takie jak krawędzie, tekstury czy bardziej złożone wzory. Filtry sterowane są hiperparametrami, które określają wielkość macierzy, wypełnienie oraz przesunięcie zależne od architektury modelu \cite{alexnet}.

Pozostałe ważne elementy tych sieci to:
warstwy łączące (ang. \textit{pooling}), które redukują wymiarowość danych, zwiększając przy tym efektywność obliczeniową i odporność na modyfikacje obrazu,
oraz warstwy w pełni połączone (ang. \emph{fully connected}) agregujące cechy wykryte przez poprzednie warstwy i służące do końcowej klasyfikacji \cite{Goodfellow-et-al-2016}.

Zaletą sieci konwolucyjnych jest automatyczne uczenie się reprezentacji cech z danych, co eliminuje konieczność ręcznego projektowania cech.

\begin{figure}[!h]
    \centering
    \includegraphics[width=0.6\linewidth]{konwolucja.png}
    \caption{Wizualizacja warstwy konwolucyjnej \cite{CNN-Explainer}}
    \label{fig:Wizualizacja warstwy konwolucyjnej}
\end{figure}

Celem jest ułożenie przestrzeni zanurzeń tak by przykłady pozytywne miały mniejszą odległość niż negatywne. Funkcja straty \textit{triplet loss} jest określana wzorem:

\begin{equation}
L(\mathbf{x}_a, \mathbf{x}_p, \mathbf{x}_n) = \max \big( 0, \, d\big(\psi(\mathbf{x}_a), \psi(\mathbf{x}_p)\big) 
- d\big(\psi(\mathbf{x}_a), \psi(\mathbf{x}_n)\big) + m \big),
\end{equation}
gdzie:
\begin{itemize}
    \item $\mathbf{x}_a$ -- tzw. kotwica (\textit{anchor}),
    \item $\mathbf{x}_p$ -- wektor pozytywny, który ma być "blisko" kotwicy
    \item $\mathbf{x}_n$ -- wektor negatywny, który ma być "daleko" kotwicy
    \item $m \in \mathbb{R}_+$ --  margines – zabezpieczenie przeciwko nakładaniu się punktów
    \item $d$ -- funkcja odległości.
\end{itemize}

Do rozpoznawania na zrzucie ekranu loga firm i pól tekstowych do wprowadzania danych autorzy wykorzystują model Faster-RCNN \cite{FasterR-CNN} modyfikując go tak aby wspólnie trenować sieć do rozpoznawania regionów i model Fast-RCNN.

\subsection{Metody miary}
metryki: Entropia Krzyżowa, F1 (binarne i wieloklasowe), ROC AUC

F1 macro averaged

\cite{grandini2020metricsmulticlassclassificationoverview}

Zadanie postawione w tej pracy jest problemem klasyfikacji wieloklasowej. Trzeba zbalansować czy chcemy skupić się na tym żeby każda klasa była dobrze odwzorowana czy żeby model częściej miał większą dokładność
 \href{https://diogoribeiro7.github.io/machine%20learning/matthew_correlation/}{github!!}

\todo[inline]{dodać ten link \textbf{to wyżej} do bib}

% % Równanie typu 'inline':
% \lipsum[2] $F = m \cdot a$ lorem ipsum dolor sit amet.
% % Równanie bez numeru
% % align oznacza wyrównanie kolejnych wierszy do '&'
% % '&' służy tylko do wyrównania i nie jest renderowany
% \begin{align*}
%     E & = mc^2 \\
%     y & = ax^2 + bx + c
% \end{align*}

% \lipsum[3]
% % Równanie numerowane: macierze
% \begin{align}
%     \begin{bmatrix}
%         1 & 0 & 0 \\
%         0 & 2 & 0 \\
%         0 & 0 & 3
%     \end{bmatrix} \cdot
%     \begin{bmatrix}
%         4 \\
%         5 \\
%         6
%     \end{bmatrix} =
%     \begin{bmatrix}
%         4  \\
%         10 \\
%         18
%     \end{bmatrix}
% \end{align}

% % Cytaty dla zapełnienia bibliografii
% \lipsum[4] Lorem ipsum dolor sit amet, consectetur adipiscing elit, sed do eiusmod tempor incididunt ut labore et dolore magna aliqua \cite{szczypiorski2015}, \cite{duqu2011}, \cite{shs2015}, \cite{wozniak2018}, \cite{dcp19}.

% % Podrozdział pierwszego poziomu
% \subsection{Critique of Pure Reason}
% \kant[1]

% % Tabela wielostronicowa, 4 kolumny
% % Kolumny typu m{} oznaczają kolumny o stałej szerokości z zawijaniem wierszy
% % Wyrównywane są domyślnie do lewej; aby ustawić inne wyrównanie,
% % stosujemy \multicolumn{1} tak jak poniżej
% \begin{longtable}{| c | m{0.58\linewidth} | r | m{0.1\linewidth} |}
%     \caption{Tabela wielostronicowa.}
%     \label{table:koszty} \\

%     \hline
%     % Nagłówek tabeli wyrównujemy do środka
%     Lp & \multicolumn{1}{c|}{Treść} & \multicolumn{1}{c|}{Kwota} & \multicolumn{1}{m{0.1\linewidth}|}{Wariant opłaty} \\ \hline\hline \endfirsthead \endfoot
%     \hline \endlastfoot

%     1  & Lorem ipsum dolor sit amet, consectetur adipiscing elit, sed do eiusmod tempor incididunt ut labore et dolore magna aliqua. & 111 111,11 zł & \multicolumn{1}{c|}{WAR1} \\ \hline
%     2  & Lorem ipsum dolor sit amet, consectetur adipiscing elit, sed do eiusmod tempor incididunt ut labore et dolore magna aliqua. & 22 222,22 zł & \multicolumn{1}{c|}{WAR1} \\ \hline
%     3  & Lorem ipsum dolor sit amet, consectetur adipiscing elit, sed do eiusmod tempor incididunt ut labore et dolore magna aliqua. & 33 333,33 zł & \multicolumn{1}{c|}{WAR1} \\ \hline
%     4  & Lorem ipsum dolor sit amet, consectetur adipiscing elit, sed do eiusmod tempor incididunt ut labore et dolore magna aliqua. & 444 444,44 zł & \multicolumn{1}{c|}{WAR1} \\ \hline
%     5  & Lorem ipsum dolor sit amet, consectetur adipiscing elit, sed do eiusmod tempor incididunt ut labore et dolore magna aliqua. & 55 555,55 zł & \multicolumn{1}{c|}{WAR1} \\ \hline
%     6  & Lorem ipsum dolor sit amet, consectetur adipiscing elit, sed do eiusmod tempor incididunt ut labore et dolore magna aliqua. & 66 666,66 zł & \multicolumn{1}{c|}{WAR1} \\ \hline
%     7  & Lorem ipsum dolor sit amet, consectetur adipiscing elit, sed do eiusmod tempor incididunt ut labore et dolore magna aliqua. & 777 777,77 zł & \multicolumn{1}{c|}{WAR1} \\ \hline
%     8  & Lorem ipsum dolor sit amet, consectetur adipiscing elit, sed do eiusmod tempor incididunt ut labore et dolore magna aliqua. & 8 888,88 zł & \multicolumn{1}{c|}{WAR1} \\ \hline
%     9  & Lorem ipsum dolor sit amet, consectetur adipiscing elit, sed do eiusmod tempor incididunt ut labore et dolore magna aliqua. & 999 999,99 zł & \multicolumn{1}{c|}{WAR1} \\ \hline
%     10 & Lorem ipsum dolor sit amet, consectetur adipiscing elit, sed do eiusmod tempor incididunt ut labore et dolore magna aliqua. & 111 111,11 zł & \multicolumn{1}{c|}{WAR2} \\ \hline
%     11 & Lorem ipsum dolor sit amet, consectetur adipiscing elit, sed do eiusmod tempor incididunt ut labore et dolore magna aliqua. & 22 222,22 zł & \multicolumn{1}{c|}{WAR2} \\ \hline
%     12 & Lorem ipsum dolor sit amet, consectetur adipiscing elit, sed do eiusmod tempor incididunt ut labore et dolore magna aliqua. & 33 333,33 zł & \multicolumn{1}{c|}{WAR2} \\ \hline
%     13 & Lorem ipsum dolor sit amet, consectetur adipiscing elit, sed do eiusmod tempor incididunt ut labore et dolore magna aliqua. & 444 444,44 zł & \multicolumn{1}{c|}{WAR2} \\ \hline
%     14 & Lorem ipsum dolor sit amet, consectetur adipiscing elit, sed do eiusmod tempor incididunt ut labore et dolore magna aliqua. & 55 555,55 zł & \multicolumn{1}{c|}{WAR2} \\ \hline
%     15 & Lorem ipsum dolor sit amet, consectetur adipiscing elit, sed do eiusmod tempor incididunt ut labore et dolore magna aliqua. & 66 666,66 zł & \multicolumn{1}{c|}{WAR2} \\ \hline
%        & \multicolumn{1}{r|}{\textbf{Suma:}} & \textbf{7 777 777,77 zł} &
% \end{longtable}

% \kant[2]

% % Nagłówki kolejnych poziomów, dla zapełnienia spisu treści
% \subsection{Caegorical Imperative} % 2.2
% \subsubsection{Deontological Ethics} % 2.2.1
% \kant[2]
% \subsubsection{Consequentialism -- the Ideal of practical reason} % 2.2.2
% \kant[3]
% \subsection{G\"odel's ontological proof} % 2.3
% \kant[9] Lorem ipsum dolor sit amet, consectetur adipiscing elit \cite{benzmuller2014}, \cite{goedel95}, \cite{wang97}, \cite{koons2005}.

% % Twierdzenia i dowody
% % Założenie
% \begin{assumption} \label{ass:1}
%     $ [\![ \ \phi \ ]\!] \Longrightarrow [\![ \ P(\phi); \neg P(\phi) \ ]\!]$
% \end{assumption}
% % Aksjomat
% \begin{axiom}[Dualność] \label{axiom:1}
%     $\neg P(\phi) \Leftrightarrow P(\neg \phi)$, równoważnie $P(\phi) \Leftrightarrow \neg P(\neg \phi)$
% \end{axiom}
% \begin{axiom}[Całkowitość] \label{axiom:2}
%     $ \left( P(\phi) \wedge \forall x: \phi(x) \Rightarrow \psi(x) \right) \Rightarrow P(\psi) $
% \end{axiom}
% \begin{axiom}[Absolutność] \label{axiom:3}
%     $ P(\phi) \Rightarrow \Box P(\phi) $
% \end{axiom}
% % Definicja
% \begin{definition} \label{def:1}
%     $ G(x) \Leftrightarrow \forall \phi: \left( P(\phi) \Rightarrow \phi(x) \right) $
% \end{definition}
% \begin{definition} \label{def:2}
%     $ \phi \ ess \ x \Leftrightarrow \phi(x) \wedge \forall \psi \left( \psi(x) \Rightarrow \Box \forall y \left( \phi(y) \Rightarrow \psi(y) \right) \right)  $
% \end{definition}
% \begin{axiom} \label{axiom:4}
%     P(G)
% \end{axiom}
% % Lemat
% \begin{lemma} \label{lemma:1}
%     $ P(\phi) \Rightarrow \Diamond \exists x : \phi(x) $
% \end{lemma}
% \begin{proof}
%     Dowód pomijamy, bo jest trywialny :)
% \end{proof}
% \begin{lemma} \label{lemma:2}
%     $ \Diamond \exists x : G(x) $
% \end{lemma}
% \begin{proof}
%     Natychmiastowy wniosek z aksjomatu \ref{axiom:4} i lematu \ref{lemma:1}.
% \end{proof}
% \begin{lemma} \label{lemma:3}
%     $ G(x) \Rightarrow G \ ess \ x $
% \end{lemma}
% \begin{proof}
%     Poprzez podstawienie do definicji \ref{def:2}.
% \end{proof}
% \begin{definition} \label{def:3}
%     $ E(x) \Leftrightarrow \forall \phi \left( \phi \ ess \ x \Rightarrow \Box\ \exists x: \phi(x) \right) $
% \end{definition}
% \begin{axiom} \label{axiom:5}
%     P(E)
% \end{axiom}
% % Twierdzenie
% \begin{theorem}
%     $ \Box\ \exists x : G(x) $
% \end{theorem}
% \begin{proof}
%     Na podstawie definicji \ref{def:1}, lematu \ref{lemma:3} i aksjomatu \ref{axiom:5}.
% \end{proof}

\clearpage % Rozdziały zaczynamy od nowej strony.
\section{Przegląd literatury}
TODO: wydzielić część o literaturze od części o sieciach itd

\cite{KnowYourPhish}

\cite{Phishpedia}
\cite{PhishZoo}
\cite{PhishWHO}

\cite{VisualPhishNet}
\cite{EMD}
\clearpage % Rozdziały zaczynamy od nowej strony.
\section{Specyfikacja wymagań i decyzje projektowe} 

W tym rozdziale zostały zdefiniowane wymagania opisujące system do rozpoznawania celu podszycia stron phishingowych.
Zostały one podzielone na dwie kategorie: wymagania funkcjonalne oraz niefunkcjonalne. 
Pierwsza z nich opisuje funkcjonalności, które system musi spełaniać definiując swoje zachowania, interakcje między nimi. 
Druga kategoria opisuje wymagania jakościowe, które system musi spełniać. 
Są to między innymi wymagania związane z wydajnością, czy jakością takowego systemu.

\subsection{Wymagania funkcjonalne}
\begin{itemize}
    \item system obsluguje zapytania za pomocą API (ang. \emph{Application Programming Interface})
    \item system zwraca odpowiedź w formacie JSON (ang. \emph{JavaScript Object Notation})
    \item system przyjmuje obraz (zrzut ekranu strony internetowej) oraz adres URL (ang. \emph{unifrom resource locator}) jako dane wejściowe
    \item system zwraca informacje o prawdopodobieństwie (pewności), że obraz przedstawia stronę phishingową
    \item system zwraca informacje o celu podszycia strony phishingowej
\end{itemize}
\subsection{Wymagania niefunkcjonalne}
\begin{itemize}    
    \item system zwraca odpowiedź w czasie krótszym niż 10 sekund
    \item system jest skonteneryzowany w oparciu o Docker
    \item system jest modularny tak by umożliwić dołożenie nowych metod klasyfikacji
    \item system jest napisany w języku Python
    \item system może być dowolnym zbiorem danych
\end{itemize}

\subsection{Architektura}
TODO: opisać diagramy
System składa się z trzech głównych modułów: API, klasyfikatora oraz baz danych.
\begin{figure}[!h]
    % Wyrównanie obrazka, szerokość i plik
    % Zamiast width można też użyć height, etc.
    \centering \includegraphics[width=0.5\linewidth]{System2.pdf}
    % Podpis umieszczamy pod obrazkiem
    % znacznik \caption służy również do wygenerowania numeru obrazka
    \caption{Diagram systemu}
    % \label pozwala odwołać się do obrazka w innych miejscach za pomocą \ref
    % odwołanie \ref renderuje się jako numer obrazka,
    % dlatego zawsze najpierw używaj \caption a potem \label
    \label{fig:diagram-systemu}
\end{figure}

% Przykładowy obrazek
\begin{figure}[!h]
    % Wyrównanie obrazka, szerokość i plik
    % Zamiast width można też użyć height, etc.
    \centering \includegraphics[width=1\linewidth]{Klasyfikator2.pdf}
    % Podpis umieszczamy pod obrazkiem
    % znacznik \caption służy również do wygenerowania numeru obrazka
    \caption{Diagram klasyfikatora}
    % \label pozwala odwołać się do obrazka w innych miejscach za pomocą \ref
    % odwołanie \ref renderuje się jako numer obrazka,
    % dlatego zawsze najpierw używaj \caption a potem \label
    \label{fig:diagram-klasyfikatora}
\end{figure}

\it{Każda metoda będzie implementować interfejs odpowiedzialny za wczytanie danych i zapisanie wyniku. Dodatkowo będzie to oddzielny kontener}

\subsection{Decyzje projektowe}
TODO: uzasadnić wybór
Traktowanie modeli jako blackbox -> phish +1 klasa (benign) (najłatwiej tak uogólnić wyniki)

\subsubsection{Wybór rozwiązań}
TODO: uzasadnić wybór
Python, 
FastAPI,
Pytorch, Tensorflow
Scikit learn
MongoDB,
MinIO

% % Fragment kodu źródłowego programu
% % \addmargin pozwala na wcięcie kodu od lewej (tu: 8mm).
% % Wcięcie służy do tego, aby numery linii nie wystawały poza lewy margines.
% % Druga liczba oznacza wcięcie od prawej.
% \begin{addmargin}[8mm]{0mm}
% \begin{lstlisting}[
%     language=HTML,
%     numbers=left,
%     firstnumber=1,
%     caption={\emph{Hello world} w HTML},
%     aboveskip=10pt
% ]
% <html>
%   <head>
%     <title>Hello world!</title>
%   </head>
%   <body>
%     Hello world!
%   </body>
% </html>
% \end{lstlisting}
% \end{addmargin}

% \lipsum[11]

% % Dla dłuższych numerów linii potrzebne jest większe wcięcie.
% \begin{addmargin}[12mm]{0mm}
% \begin{lstlisting}[
%     language=C++,
%     numbers=left,
%     firstnumber=147,
%     caption={Generowanie sekwencji Collatza w języku C++},
%     aboveskip=10pt
% ]
% class Collatz {
%   private:
%     unsigned current_val_;
%     void update_val() {
%         if( current_val_ % 2 == 0 )
%             current_val_ /= 2;
%         else
%             current_val_ = current_val_ * 3 + 1;
%     }

%   public:
%     explicit Collatz(unsigned initial_value) :
%         current_val_(initial_value) {}
%     void print_sequence() {
%         unsigned i = 1;
%         while( current_val_ > 1 ) {
%             std::cout
%                 << "val " << i << " = " << current_val_
%                 << std::endl;
%             update_val(); ++i;
%         }
%     }
% };

% int main() {
%   // prints Collatz seqence, starting from 194375
%   Collatz seq(194375);
%   seq.print_sequence();
%   return 0;
% }
% \end{lstlisting}
% \end{addmargin}

% \lipsum[12]
\clearpage
\section{Implementacja}
cele:
- trackowanie ekseprymentów (wandb)
- łatwe uruchamianie (skrypty)
- ???

W trakcie implementacji napotkano różne problemy, pogrupowane ze względu na metodę, z którą są związane.

\todo[inline]{vp na multigpu}


\subsection{Środowisko}
\subsection{Integracja nowych zbiorów danych}
\subsection{Testy}

\todo[inline]{w którym rozdziale dać metody które testowałem?}
\subsection{Porównywanie wizualne}
\subsubsection{EMD}
\subsubsection{Phash}

\subsection{Porównywanie wizualne}
\subsubsection{deskryptory wizualne}
\subsubsection{bovw}

\subsection{Sieci neuronowe}
\subsubsection{VisualPhishNet}

\todo[inline]{opisać czym są notataniki Jupyter i dlaczego są gorsze niż python}.
\todo[inline]{cite logging - https://docs.python.org/3/library/logging.html}
\todo[inline]{cite wandb}

Autorzy danej pracy udostępnili kod w postaci notatników Jupyter. Zawierają one notatniki z procesem treningu a także notatki z procesami ewaluacji jak i interferencji. \textbf{wstawka}. Z tych powodów zdecydowano się na przepisanie rozwiązań w postaci kodu w języku Python. Dodatkowo rozszerzono to rozwiązanie o informacje pomocnicze wykorzystujące bibliotekę \textbf{logging} oraz śledzenie eksperymentów wykorzystując platformę \textbf{wandb.ai}. Dzięki temu ułatwiono wersjonowanie modeli ich treningu i ewaluacji. Ponadto uproszczono wizualizację i śledzenie eksperymentów.

W trakcie przeprowadzania eksperymentów napotkano problemy, które rozwiązano za pomocą częstszego zapisywania wyników pośrednich oraz szybszej weryfikacji poprawności metod uczących, korzystając z mniejszego zbioru treningowego. Dodatkowo, zapisywano przetransformowane dane uczące, by w razie awarii, szybciej wznowić eksperymenty.

\subsubsection{Phishpedia}
Natomiast autorzy tej pracy, udostępnili swoje rozwiązanie w postaci skryptów języka Python. Jednakże posiadają one wyłącznie skrypty do interferencji modelu. Z tego względu zaimplementowano skrypty do ewaluacji wyników oraz treningu modeli bazowych. System Phishpedia nie wymaga treningu co zostało wspomniane w \textbf{wstawka}

\todo[inline]{zaimplementować treningu modelu z logo2k+}
\todo[inline]{phishpedia nie wymaga treningu - dodać odniesienie przy opisie literatury}

Podczas ewaluacji napotkano problemy z niespójnym kodowaniem plików w zbiorach ewaluacyjnych oraz brakującymi danymi. Przykładowo, plik \textbf{domain\_map.pkl} nie zawierał wszystkich stron chronionych. Objawiało się to logotypami, które chroniono, lecz brakowało domen tych podmiotów. Dlatego zaimplementowano normalizację danych tekstowych oraz uzupełnianie brakujących danych w skryptach \textbf{wstawka}

\todo[inline]{skrytpy z nazwy\: convert\_encodings, find\_missing_files, fix\_script}.

\clearpage
\section{Metodyka badawcza i badania eksperymentalne}

W tym rozdziale zostaną opisane metody oceny konkretnych rozwiązań. Udzielono również odpowiedzi na pytanie, jakich zbiorów danych użyto? Jakie metryki zostały wykorzystane, a także jakie są wyniki poszczególnych modeli? Ponadto zostanie poruszona tematyka dalszego rozwoju systemu.

\subsection{Metodyka badawcza}
W ramach tej pracy skupiono się na dwóch metodach "zaawansowanych" opisanych szczegółowo we wcześniejszych rozdziałach (\textit{VisualPhishNet} i \textit{Phishpedia}) oraz zestawiono je z metodą bazową.
\todo[inline]{nadać temu jakąś nazwę}
Żeby móc ocenić je między sobą, nie tylko na zbiorach danych przygotowanych przez autorów tych prac, postanowiono dokonać predykcji parami każdego zbioru i metody. Rozwiązania wymienione wyżej traktują zadanie rozpoznania phishingu jako zadanie klasyfikacji binarnej, czy dana strona jest prawdziwa? Natomiast założeniem tej pracy jest klasyfikacja wieloklasowa. Nie tylko czy strona jest prawdziwa ale też jak jest strona fałszywa to pod jaki cel się podszywa. Wobec tego pierwszym założeniem jest klasyfikacja wieloklasowa z $N+1$ klasami, gdzie $N$ oznacza liczbę podmiotów chronionych a dodatkowa klasa opisuje klasę prawdziwą z ang \textit{benign}.


Z racji różnych wyjść metod, \todo[inline]{vp daje odległości, pp daje 0/1 i pewność sieci syjamskich. Phash daje odległość} zdecydowano się zastosować miarę entropii krzyżowej. Dodatkowo by móc łatwiej porównywać metody binarne F1 do określenia, czy strona jest prawdziwa. Na potrzeby wizualizacji użyto miary ROC AUC
\todo[inline]{czy uzyc wieloklasowej F1?}

TODO: opis w jakim setupie, na jakich datasetach, posługując się jakimi metrykami i w jaki sposób zostały przeprowadzone badania

\subsection{Ocena jakości poszczególnych modeli}
tj. testowanie każdego modelu na każdym zbiorze danych

model / dane | pp | vp | bovw | phash |
pp |
vp |
bovw |
phash |


metryki: f1, precision, recall
zbior certu jako zbiór testowy
\subsection{Ocena całościowa klasyfikatora}
\subsection{Dalsze kierunki rozwoju}

\clearpage
\section{Podsumowanie}

\begin{quote}
    Security is a process, not a product\cite{schneier_security_process}
\end{quote}
% TODO: jakie zbiory danych?
Zbiory danych są trudne do stworzenia i utrzymania. Wymaga to wiele pracy i środków.
Ponadto wraz z rozwojem metod detekcji ataków phishingowych, atakujący wykorzystują jeszcze sprytniejsze sposoby żeby uniknąć wykrycia. Z tego powodu nie należy się zniechęcać i dalej pracować nad rozwojem metod walki z cyberprzestępczością w tym z phishingiem, by zabezpieczać społeczeństwo przed oszustwami skierowanymi przeciwko niemu.

\subsection{Wnioski}
\subsection{Możliwości rozwoju}
\subsection{Zakończenie}

%---------------
% Bibliografia
%---------------
\cleardoublepage % Zaczynamy od nieparzystej strony
\printbibliography
\clearpage

% Wykaz symboli i skrótów.
% Pamiętaj, żeby posortować symbole alfabetycznie
% we własnym zakresie. Makro \acronymlist
% generuje właściwy tytuł sekcji, w zależności od języka.
% Makro \acronym dodaje skrót/symbol do listy,
% zapewniając podstawowe formatowanie.
\acronymlist
\acronym{API}{ang. \emph{Application Programming Interface}}
\acronym{EiTI}{Wydział Elektroniki i Technik Informacyjnych}
\acronym{JSON}{ang. \emph{JavaScript Object Notation}}
\acronym{PW}{Politechnika Warszawska}
\acronym{URL}{ang. \emph{unifrom resource locator}}
\vspace{0.8cm}

%--------------------------------------
% Spisy: rysunków, tabel, załączników
%--------------------------------------
\pagestyle{plain}

\listoffigurestoc    % Spis rysunków.
\vspace{1cm}         % vertical space
\listoftablestoc     % Spis tabel.
\vspace{1cm}         % vertical space
\renewcommand\lstlistlistingname{Lista listingów}
\lstlistoflistings
\vspace{1cm}         % vertical space
\listofappendicestoc % Spis załączników

%-------------
% Załączniki
%-------------

% Obrazki i tabele w załącznikach nie trafiają do spisów
% \captionsetup[figure]{list=no}
% \captionsetup[table]{list=no}

% % Załącznik 1
% \clearpage
% \appendix{Nazwa załącznika 1}
% \lipsum[1-3]
% \begin{figure}[!h]
% 	\centering \includegraphics[width=0.5\linewidth]{logopw2.png}
% 	\caption{Obrazek w załączniku.}
% \end{figure}
% \lipsum[4-7]

% % Załącznik 2
% \clearpage
% \appendix{Nazwa załącznika 2}
% \lipsum[1-2]
% \begin{table}[!h] \centering
%     \caption{Tabela w załączniku.}
%     \begin{tabular} {| c | c | r |} \hline
%         Kolumna 1       & Kolumna 2 & Liczba \\ \hline\hline
%         cell1           & cell2     & 60     \\ \hline
%         \multicolumn{2}{|r|}{Suma:} & 123,45 \\ \hline
%     \end{tabular}
% \end{table}
% \lipsum[3-4]

% Używając powyższych spisów jako szablonu,
% możesz dodać również swój własny wykaz,
% np. spis algorytmów.
\listoftodos
\end{document} % Dobranoc.
