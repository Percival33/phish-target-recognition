\clearpage % Rozdziały zaczynamy od nowej strony.
\section{Podstawy teoretyczne}
W tym rozdziale scharakteryzowano podstawy teoretyczne zagrożenia typu phishing jak i pojęć związanych z automatycznycmi systemami, które są wykorzystywane w pracy.
\todo[inline]{uzupełnić wstęp teoretyczny potrzebny do zrozumienia tematyki - co to jest phishing, jakie są metody jego rozpoznawania - pojęcia, pomysły, sieci konowucyjne, sieci syjamskie + transfer learning, deskryptory wizualne, hasze wizualne}

\noindent\makebox[\linewidth]{\rule{\paperwidth}{0.4pt}}

W tym rozdziale zostaną omówione zagadnienia związane z stronami phishingowymi, ochronie iternautów oraz inne zagadnienie teoretyczne potrzebne do zrozumienia tematu. 

istniejące sposoby rozpoznawania stron phishingowych na podstawie różnych atrybutów. 
Przytoczone rozwiązania określają czy strona internetowa jest stroną phishingową, korzystając z cech takich jak analiza kodu strony, adresu URL czy też rozpoznawanie wizualne. W dalszej części tego rozdziału zostanie omówiona tematyka zbiorów danych. 
\subsection{Phishing}
- co to jest
- przykłady
- kto jest celem

\todo[inline]{dodać jakieś inforamcje jak to wygląda w innych krajach np albo jakiś rys historyczny :xD: tak żeby wspomnieć na wstępie o tym że jest to współpraca z CERT Polska}


\subsection{CERT Polska}
- co to jest CERT Polska
- czym się zajmuje i jaki to ma związek z phishingiem
- jakie są inne instytucje zajmujące się tym problemem
- strona gdzie można zgłaszać informacje o phishingu

pełniącego obowiązki CSIRT\footnote{\textit{Computer Security Incident Response Team} - Zespół Reagowania na Incydenty Bezpieczeństwa Komputerowego}  NASK\cite{certpl}

Do tej pory stosowano różne sposoby zapobiegania oszustwom, w których stosowano podejścia blacklist \cite{google-blacklist}, (również CERT Polska udostępnia listę\footnote{\href{https://cert.pl/lista-ostrzezen/}{https://cert.pl/lista-ostrzezen/}} podejrzanych domen). 
Oprócz tego wykorzystywane są rozwiązania opierające się na kodzie HTML danej strony np. \cite{PhishZoo}, jednak rozwiązania te są mało efektywne, ponieważ podobne wizualnie strony mogą mieć zupełnie inny kod źródłowy. 

\subsection{Kod źródłowy}
Istnieje wiele rozwiązań, które próbują określić cel ataku na podstawie adresu URL i jego atrybutów. W pracy \cite{KnowYourPhish} autorzy opierają się na słowach kluczowych występujących na stronie i porównaniu z nazwą podmiotu wynikającego z adresu URL na podstawie \textit{fqdn}
\footnote{fully qualified domain name -  pełna, jednoznaczna nazwa domenowa, określająca położenie danego węzła w systemie DNS. - \href{https://pl.wikipedia.org/wiki/Fully_Qualified_Domain_Name}{https://pl.wikipedia.org/wiki/Fully\_Qualified\_Domain\_Name}}.
\begin{figure}[!h]
    \centering
    \includegraphics[width=0.5\linewidth]{FQDN-example.png}
    \caption{Struktura adresu URL z pracy \cite{KnowYourPhish}}
    \label{fig:Struktura adresu URL z pracy}
\end{figure}

gdzie:
\begin{itemize}
    \item protocol = https
    \item FQDN = www.amazon.co.uk
    \item RDN = amazon.co.uk
    \item mld = amazon
    \item FreeURL = {www, /ap/signin? encoding=UTF8}
\end{itemize}

Inne prace \cite{PhishWHO} wykorzystują słowa kluczowe i wyszukiwarkę internetową porównując najwyżej notowane wyniki przy wyszukaniu słów kluczowych na danej stronie by w ten sposób określić cel podszycia strony. Kolejnym podejściem jest analiza linków wychodzących z badanej strony i sprawdzenie czy prowadzą na nią z powrotem, budując w ten sposób graf. Na jego podstawie określany jest cel podszycia jednocześnie stwierdzając czy strona jest prawdziwa.

\begin{figure}[!h]
    \centering
    \includegraphics[width=0.8\linewidth]{Antiphishing through phishing discovery.png}
    \caption{Graf zbudowany z linków dostępnych na badanej stronie z \cite{PhishWHO}}
    \label{fig:enter-label}
\end{figure}

\subsection{Hasze wizualne i deskryptory}

W zastosowaniach kryptograficznych do jednoznacznej identyfikacji pliku wykorzystywane są funkcje skrótu takie jak: MD5, SHA-1, SHA-2. Dla wejścia o dowolnym rozmiarze uzyskujemy skrót stałego rozmiaru.
Zależy nam na minimalizacji liczby kolizji, czyli przypadków gdy otrzymano taki sam skrót dla różnych danych wejściowych.

Do wykrywania podobnych informacji by wykrywać naruszenia praw autorskich lub wykrywać nielegalne treści, wykorzystuje się technikę zwaną \textit{Locality-sensitive hash}, które mają na celu pogrupowanie elementów tak, że podobne do siebie trafiają do jednej grupy. W tym celu, przeciwnie niż do zastosowań kryptograficznych maksymalizowana jest liczba kolizji. Tak aby bardziej podobne dane wejściowe uzyskały bliższe sobie wartości skrótu, uzyskując przy tym odcisk palca - \textit{fingerprint} \cite{Zauner2010ImplementationAB}.

\todo[inline]{czym są deskryptory?}
\todo[inline]{co to są zanurzenia}
\subsection{Sieci neuronowe}
Z racji że rozwiązania opierające się wyłącznie na rozpoznawaniu stron na podstawie kolorów \cite{EMD} jest mało skuteczne, autorzy prac skupiają się na zastosowaniu sieci głębokich a dokładniej sieci konwolucyjnych.

\todo[inline]{opisać sieci syjamnskie}

Sieć konwolucyjna (CNN) jest to klasa sztucznych sieci neuronowych, najczęściej wykorzystywana w zadaniach wizji komputerowej, ze względu na efektywne przetwarzanie danych o strukturze przestrzennej. Kluczowym elementem sieci są warstwy konwolucyjne. Każda z nich zawiera zestaw filtrów, które wykonują operację konwolucji na danych wejściowych. Filtry te uczą się rozpoznawać określone cechy, takie jak krawędzie, tekstury czy bardziej złożone wzory. Filtry sterowane są hiperparametrami, które określają wielkość macierzy, wypełnienie oraz przesunięcie zależne od architektury modelu \cite{alexnet}.

Pozostałe ważne elementy tych sieci to:
warstwy łączące (ang. \textit{pooling}), które redukują wymiarowość danych, zwiększając przy tym efektywność obliczeniową i odporność na modyfikacje obrazu,
oraz warstwy w pełni połączone (ang. \emph{fully connected}) agregujące cechy wykryte przez poprzednie warstwy i służące do końcowej klasyfikacji \cite{Goodfellow-et-al-2016}.

Zaletą sieci konwolucyjnych jest automatyczne uczenie się reprezentacji cech z danych, co eliminuje konieczność ręcznego projektowania cech.

\begin{figure}[!h]
    \centering
    \includegraphics[width=0.6\linewidth]{konwolucja.png}
    \caption{Wizualizacja warstwy konwolucyjnej \cite{CNN-Explainer}}
    \label{fig:Wizualizacja warstwy konwolucyjnej}
\end{figure}

Celem jest ułożenie przestrzeni zanurzeń tak by przykłady pozytywne miały mniejszą odległość niż negatywne. Funkcja straty \textit{triplet loss} jest określana wzorem:

\begin{equation}
L(\mathbf{x}_a, \mathbf{x}_p, \mathbf{x}_n) = \max \big( 0, \, d\big(\psi(\mathbf{x}_a), \psi(\mathbf{x}_p)\big) 
- d\big(\psi(\mathbf{x}_a), \psi(\mathbf{x}_n)\big) + m \big),
\end{equation}
gdzie:
\begin{itemize}
    \item $\mathbf{x}_a$ -- tzw. kotwica (\textit{anchor}),
    \item $\mathbf{x}_p$ -- wektor pozytywny, który ma być "blisko" kotwicy
    \item $\mathbf{x}_n$ -- wektor negatywny, który ma być "daleko" kotwicy
    \item $m \in \mathbb{R}_+$ --  margines – zabezpieczenie przeciwko nakładaniu się punktów
    \item $d$ -- funkcja odległości.
\end{itemize}

Do rozpoznawania na zrzucie ekranu loga firm i pól tekstowych do wprowadzania danych autorzy wykorzystują model Faster-RCNN \cite{FasterR-CNN} modyfikując go tak aby wspólnie trenować sieć do rozpoznawania regionów i model Fast-RCNN.

\subsection{Metody miary}
metryki: Entropia Krzyżowa, F1 (binarne i wieloklasowe), ROC AUC

F1 macro averaged

\cite{grandini2020metricsmulticlassclassificationoverview}

Zadanie postawione w tej pracy jest problemem klasyfikacji wieloklasowej. Trzeba zbalansować czy chcemy skupić się na tym żeby każda klasa była dobrze odwzorowana czy żeby model częściej miał większą dokładność
 \href{https://diogoribeiro7.github.io/machine%20learning/matthew_correlation/}{github!!}

\todo[inline]{dodać ten link \textbf{to wyżej} do bib}

% % Równanie typu 'inline':
% \lipsum[2] $F = m \cdot a$ lorem ipsum dolor sit amet.
% % Równanie bez numeru
% % align oznacza wyrównanie kolejnych wierszy do '&'
% % '&' służy tylko do wyrównania i nie jest renderowany
% \begin{align*}
%     E & = mc^2 \\
%     y & = ax^2 + bx + c
% \end{align*}

% \lipsum[3]
% % Równanie numerowane: macierze
% \begin{align}
%     \begin{bmatrix}
%         1 & 0 & 0 \\
%         0 & 2 & 0 \\
%         0 & 0 & 3
%     \end{bmatrix} \cdot
%     \begin{bmatrix}
%         4 \\
%         5 \\
%         6
%     \end{bmatrix} =
%     \begin{bmatrix}
%         4  \\
%         10 \\
%         18
%     \end{bmatrix}
% \end{align}

% % Cytaty dla zapełnienia bibliografii
% \lipsum[4] Lorem ipsum dolor sit amet, consectetur adipiscing elit, sed do eiusmod tempor incididunt ut labore et dolore magna aliqua \cite{szczypiorski2015}, \cite{duqu2011}, \cite{shs2015}, \cite{wozniak2018}, \cite{dcp19}.

% % Podrozdział pierwszego poziomu
% \subsection{Critique of Pure Reason}
% \kant[1]

% % Tabela wielostronicowa, 4 kolumny
% % Kolumny typu m{} oznaczają kolumny o stałej szerokości z zawijaniem wierszy
% % Wyrównywane są domyślnie do lewej; aby ustawić inne wyrównanie,
% % stosujemy \multicolumn{1} tak jak poniżej
% \begin{longtable}{| c | m{0.58\linewidth} | r | m{0.1\linewidth} |}
%     \caption{Tabela wielostronicowa.}
%     \label{table:koszty} \\

%     \hline
%     % Nagłówek tabeli wyrównujemy do środka
%     Lp & \multicolumn{1}{c|}{Treść} & \multicolumn{1}{c|}{Kwota} & \multicolumn{1}{m{0.1\linewidth}|}{Wariant opłaty} \\ \hline\hline \endfirsthead \endfoot
%     \hline \endlastfoot

%     1  & Lorem ipsum dolor sit amet, consectetur adipiscing elit, sed do eiusmod tempor incididunt ut labore et dolore magna aliqua. & 111 111,11 zł & \multicolumn{1}{c|}{WAR1} \\ \hline
%     2  & Lorem ipsum dolor sit amet, consectetur adipiscing elit, sed do eiusmod tempor incididunt ut labore et dolore magna aliqua. & 22 222,22 zł & \multicolumn{1}{c|}{WAR1} \\ \hline
%     3  & Lorem ipsum dolor sit amet, consectetur adipiscing elit, sed do eiusmod tempor incididunt ut labore et dolore magna aliqua. & 33 333,33 zł & \multicolumn{1}{c|}{WAR1} \\ \hline
%     4  & Lorem ipsum dolor sit amet, consectetur adipiscing elit, sed do eiusmod tempor incididunt ut labore et dolore magna aliqua. & 444 444,44 zł & \multicolumn{1}{c|}{WAR1} \\ \hline
%     5  & Lorem ipsum dolor sit amet, consectetur adipiscing elit, sed do eiusmod tempor incididunt ut labore et dolore magna aliqua. & 55 555,55 zł & \multicolumn{1}{c|}{WAR1} \\ \hline
%     6  & Lorem ipsum dolor sit amet, consectetur adipiscing elit, sed do eiusmod tempor incididunt ut labore et dolore magna aliqua. & 66 666,66 zł & \multicolumn{1}{c|}{WAR1} \\ \hline
%     7  & Lorem ipsum dolor sit amet, consectetur adipiscing elit, sed do eiusmod tempor incididunt ut labore et dolore magna aliqua. & 777 777,77 zł & \multicolumn{1}{c|}{WAR1} \\ \hline
%     8  & Lorem ipsum dolor sit amet, consectetur adipiscing elit, sed do eiusmod tempor incididunt ut labore et dolore magna aliqua. & 8 888,88 zł & \multicolumn{1}{c|}{WAR1} \\ \hline
%     9  & Lorem ipsum dolor sit amet, consectetur adipiscing elit, sed do eiusmod tempor incididunt ut labore et dolore magna aliqua. & 999 999,99 zł & \multicolumn{1}{c|}{WAR1} \\ \hline
%     10 & Lorem ipsum dolor sit amet, consectetur adipiscing elit, sed do eiusmod tempor incididunt ut labore et dolore magna aliqua. & 111 111,11 zł & \multicolumn{1}{c|}{WAR2} \\ \hline
%     11 & Lorem ipsum dolor sit amet, consectetur adipiscing elit, sed do eiusmod tempor incididunt ut labore et dolore magna aliqua. & 22 222,22 zł & \multicolumn{1}{c|}{WAR2} \\ \hline
%     12 & Lorem ipsum dolor sit amet, consectetur adipiscing elit, sed do eiusmod tempor incididunt ut labore et dolore magna aliqua. & 33 333,33 zł & \multicolumn{1}{c|}{WAR2} \\ \hline
%     13 & Lorem ipsum dolor sit amet, consectetur adipiscing elit, sed do eiusmod tempor incididunt ut labore et dolore magna aliqua. & 444 444,44 zł & \multicolumn{1}{c|}{WAR2} \\ \hline
%     14 & Lorem ipsum dolor sit amet, consectetur adipiscing elit, sed do eiusmod tempor incididunt ut labore et dolore magna aliqua. & 55 555,55 zł & \multicolumn{1}{c|}{WAR2} \\ \hline
%     15 & Lorem ipsum dolor sit amet, consectetur adipiscing elit, sed do eiusmod tempor incididunt ut labore et dolore magna aliqua. & 66 666,66 zł & \multicolumn{1}{c|}{WAR2} \\ \hline
%        & \multicolumn{1}{r|}{\textbf{Suma:}} & \textbf{7 777 777,77 zł} &
% \end{longtable}

% \kant[2]

% % Nagłówki kolejnych poziomów, dla zapełnienia spisu treści
% \subsection{Caegorical Imperative} % 2.2
% \subsubsection{Deontological Ethics} % 2.2.1
% \kant[2]
% \subsubsection{Consequentialism -- the Ideal of practical reason} % 2.2.2
% \kant[3]
% \subsection{G\"odel's ontological proof} % 2.3
% \kant[9] Lorem ipsum dolor sit amet, consectetur adipiscing elit \cite{benzmuller2014}, \cite{goedel95}, \cite{wang97}, \cite{koons2005}.

% % Twierdzenia i dowody
% % Założenie
% \begin{assumption} \label{ass:1}
%     $ [\![ \ \phi \ ]\!] \Longrightarrow [\![ \ P(\phi); \neg P(\phi) \ ]\!]$
% \end{assumption}
% % Aksjomat
% \begin{axiom}[Dualność] \label{axiom:1}
%     $\neg P(\phi) \Leftrightarrow P(\neg \phi)$, równoważnie $P(\phi) \Leftrightarrow \neg P(\neg \phi)$
% \end{axiom}
% \begin{axiom}[Całkowitość] \label{axiom:2}
%     $ \left( P(\phi) \wedge \forall x: \phi(x) \Rightarrow \psi(x) \right) \Rightarrow P(\psi) $
% \end{axiom}
% \begin{axiom}[Absolutność] \label{axiom:3}
%     $ P(\phi) \Rightarrow \Box P(\phi) $
% \end{axiom}
% % Definicja
% \begin{definition} \label{def:1}
%     $ G(x) \Leftrightarrow \forall \phi: \left( P(\phi) \Rightarrow \phi(x) \right) $
% \end{definition}
% \begin{definition} \label{def:2}
%     $ \phi \ ess \ x \Leftrightarrow \phi(x) \wedge \forall \psi \left( \psi(x) \Rightarrow \Box \forall y \left( \phi(y) \Rightarrow \psi(y) \right) \right)  $
% \end{definition}
% \begin{axiom} \label{axiom:4}
%     P(G)
% \end{axiom}
% % Lemat
% \begin{lemma} \label{lemma:1}
%     $ P(\phi) \Rightarrow \Diamond \exists x : \phi(x) $
% \end{lemma}
% \begin{proof}
%     Dowód pomijamy, bo jest trywialny :)
% \end{proof}
% \begin{lemma} \label{lemma:2}
%     $ \Diamond \exists x : G(x) $
% \end{lemma}
% \begin{proof}
%     Natychmiastowy wniosek z aksjomatu \ref{axiom:4} i lematu \ref{lemma:1}.
% \end{proof}
% \begin{lemma} \label{lemma:3}
%     $ G(x) \Rightarrow G \ ess \ x $
% \end{lemma}
% \begin{proof}
%     Poprzez podstawienie do definicji \ref{def:2}.
% \end{proof}
% \begin{definition} \label{def:3}
%     $ E(x) \Leftrightarrow \forall \phi \left( \phi \ ess \ x \Rightarrow \Box\ \exists x: \phi(x) \right) $
% \end{definition}
% \begin{axiom} \label{axiom:5}
%     P(E)
% \end{axiom}
% % Twierdzenie
% \begin{theorem}
%     $ \Box\ \exists x : G(x) $
% \end{theorem}
% \begin{proof}
%     Na podstawie definicji \ref{def:1}, lematu \ref{lemma:3} i aksjomatu \ref{axiom:5}.
% \end{proof}
