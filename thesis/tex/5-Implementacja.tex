\clearpage
\section{Implementacja}
cele:
- trackowanie ekseprymentów (wandb)
- łatwe uruchamianie (skrypty)
- ???

W trakcie implementacji napotkano różne problemy, pogrupowane ze względu na metodę, z którą są związane.

\todo[inline]{vp na multigpu}


\subsection{Środowisko}
\subsection{Integracja nowych zbiorów danych}
\subsection{Testy}

\todo[inline]{w którym rozdziale dać metody które testowałem?}
\subsection{Porównywanie wizualne}
\subsubsection{EMD}
\subsubsection{Phash}

\subsection{Porównywanie wizualne}
\subsubsection{deskryptory wizualne}
\subsubsection{bovw}

\subsection{Sieci neuronowe}
\subsubsection{VisualPhishNet}

\todo[inline]{opisać czym są notataniki Jupyter i dlaczego są gorsze niż python}.
\todo[inline]{cite logging - https://docs.python.org/3/library/logging.html}
\todo[inline]{cite wandb}

Autorzy danej pracy udostępnili kod w postaci notatników Jupyter. Zawierają one notatniki z procesem treningu a także notatki z procesami ewaluacji jak i interferencji. \textbf{wstawka}. Z tych powodów zdecydowano się na przepisanie rozwiązań w postaci kodu w języku Python. Dodatkowo rozszerzono to rozwiązanie o informacje pomocnicze wykorzystujące bibliotekę \textbf{logging} oraz śledzenie eksperymentów wykorzystując platformę \textbf{wandb.ai}. Dzięki temu ułatwiono wersjonowanie modeli ich treningu i ewaluacji. Ponadto uproszczono wizualizację i śledzenie eksperymentów.

W trakcie przeprowadzania eksperymentów napotkano problemy, które rozwiązano za pomocą częstszego zapisywania wyników pośrednich oraz szybszej weryfikacji poprawności metod uczących, korzystając z mniejszego zbioru treningowego. Dodatkowo, zapisywano przetransformowane dane uczące, by w razie awarii, szybciej wznowić eksperymenty.

\subsubsection{Phishpedia}
Natomiast autorzy tej pracy, udostępnili swoje rozwiązanie w postaci skryptów języka Python. Jednakże posiadają one wyłącznie skrypty do interferencji modelu. Z tego względu zaimplementowano skrypty do ewaluacji wyników oraz treningu modeli bazowych. System Phishpedia nie wymaga treningu co zostało wspomniane w \textbf{wstawka}

\todo[inline]{zaimplementować treningu modelu z logo2k+}
\todo[inline]{phishpedia nie wymaga treningu - dodać odniesienie przy opisie literatury}

Podczas ewaluacji napotkano problemy z niespójnym kodowaniem plików w zbiorach ewaluacyjnych oraz brakującymi danymi. Przykładowo, plik \textbf{domain\_map.pkl} nie zawierał wszystkich stron chronionych. Objawiało się to logotypami, które chroniono, lecz brakowało domen tych podmiotów. Dlatego zaimplementowano normalizację danych tekstowych oraz uzupełnianie brakujących danych w skryptach \textbf{wstawka}

\todo[inline]{skrytpy z nazwy\: convert\_encodings, find\_missing_files, fix\_script}.
