\clearpage % Rozdziały zaczynamy od nowej strony.
\section{Wstęp teoretyczny}
Pośród istniejących rozwiązań określających czy strona internetowa jest stroną phishingową czy nie, wyłaniają się główne motywy skupiające się na analizie kodu strony, adresu URL czy też rozpoznawania wizualnego

\subsection{Kod źródłowy}
Istnieje wiele rozwiązań, które próbują określić cel ataku na podstawie adresu URL i jego atrybutów. W pracy \cite{KnowYourPhish} autorzy opierają się na słowach kluczowych występujących na stronie i porównaniu z nazwą podmiotu wynikającego z adresu URL na podstawie \textit{fqdn}
\footnote{fully qualified domain name -  pełna, jednoznaczna nazwa domenowa, określająca położenie danego węzła w systemie DNS. - \href{https://pl.wikipedia.org/wiki/Fully_Qualified_Domain_Name}{https://pl.wikipedia.org/wiki/Fully\_Qualified\_Domain\_Name}}.
\begin{figure}
    \centering
    \includegraphics[width=0.5\linewidth]{FQDN-example.png}
    \caption{Struktura adresu URL z pracy \cite{KnowYourPhish}}
    \label{fig:Struktura adresu URL z pracy}
\end{figure}

gdzie:
\begin{itemize}
    \item protocol = https
    \item FQDN = www.amazon.co.uk
    \item RDN = amazon.co.uk
    \item mld = amazon
    \item FreeURL = {www, /ap/signin? encoding=UTF8}
\end{itemize}

Inne prace \cite{PhishWHO} wykorzystują słowa kluczowe i wyszukiwarkę internetową porównując najwyżej listowane wyniki przy wyszukaniu słów kluczowych na danej stronie by w ten sposób określić cel podszycia strony. Kolejnym podejściem jest analiza linków wychodzących z badanej strony i sprawdzenie czy prowadzą na nią z powrotem, budując w ten sposób graf. Na jego podstawie określany jest cel podszycia jednocześnie stwierdzając czy strona jest prawdziwa.

\begin{figure}
    \centering
    \includegraphics[width=0.8\linewidth]{Antiphishing through phishing discovery.png}
    \caption{Graf zbudowany z linków dostępnych na badanej stronie z \cite{PhishWHO}}
    \label{fig:enter-label}
\end{figure}

\subsection{Hasze wizualne i deskryptory}
\subsubsection{Hasze wizualne}

W zastosowaniach kryptograficznych do jednoznacznej identyfikacji pliku wykorzystywane są funkcje skrótu takie jak: MD5, SHA-1, SHA-2. Dla wejścia o dowolnym rozmiarze uzyskujemy skrót stałego rozmiaru.
Zależy nam na minimalizacji liczby kolizji, czyli przypadków gdy otrzymano taki sam skrót dla różnych danych wejściowych.

Do wykrywania podobnych informacji by wykrywać naruszenia praw autorskich lub wykrywać nielegalne treści, wykorzystuje się technikę zwaną \textit{Locality-sensitive hash}, które mają na celu pogrupowanie elementów tak, że podobne do siebie trafiają do jednej grupy. W tym celu, przeciwnie niż do zastosowań kryptograficznych maksymalizowana jest liczba kolizji. Tak aby bardziej podobne dane wejściowe uzyskały bliższe sobie wartości skrótu, uzyskując przy tym odcisk palca - \textit{fingerprint} \cite{Zauner2010ImplementationAB}.

W pracy\cite{EMD} zastosowano porównywanie wizualne stron, które polega na równywaniu zrzutów ekranu, zmniejszając je i obliczając podobieństwo korzystając z metryki \textit{Earth Mover’s Distance}.

\subsubsection{Deskryptory wizualne}
Innym podejściem, wartym uwagi są rozwiązania ekstrachujące deskryptory wizualne \cite{Daisy}. Deskryptory wizualne są to histogramy gradientów otaczających punkt na obrazie. Oprócz tego wykorzystywany jest również algorytm SIFT (Scale Invariant Feature
Transform)\cite{PhishZoo}. Przekształca on zdjęcia w zbiór wektorów cech i na tej podstawie porównuje loga firm w celu rozpoznania podszywanej strony.
\begin{figure}
    \centering
    \includegraphics[width=0.6\linewidth]{sphx_glr_plot_daisy_001.png}
    \caption{Deskryptory DAISY \cite{scikit-image-daisy}}
    \label{fig:Deskryptory DAISY}
\end{figure}

Oba te rozwiązania później są wykorzystywane do zastosowań \textit{bag of features}, które polegają na klastrowaniu otrzymanych deskryptorów by utworzyć reprezentację obrazów i porównywać nie same obrazy a deskryptory obliczone na nich. 

\subsection{Sieci neuronowe}
Z racji że rozwiązania opierające się wyłącznie na rozpoznawaniu stron na podstawie kolorów \cite{EMD} jest mało skuteczne, autorzy prac skupiają się na zastosowaniu sieci głębokich a dokładniej sieci konwolucyjnych.


Sieć konwolucyjna (CNN) jest to klasa sztucznych sieci neuronowych, najczęściej wykorzystywana w zadaniach wizji komputerowej, ze względu na efektywne przetwarzanie danych o strukturze przestrzennej. Kluczowym elementem sieci są warstwy konwolucyjne. Każda z nich zawiera zestaw filtrów, które wykonują operację konwolucji na danych wejściowych. Filtry te uczą się rozpoznawać określone cechy, takie jak krawędzie, tekstury czy bardziej złożone wzory. Filtry sterowane są hiperparametrami, które określają wielkość macierzy, wypełnienie oraz przesunięcie zależne od architektury modelu.\cite{alexnet}

Pozostałe ważne elementy tych sieci to:
warstwy łączące (\textit{pooling}), które redukują wymiarowość danych, zwiększając przy tym efektywność obliczeniową i odporność na modyfikacje obrazu,
oraz warstwy w pełni połączone (\textit{fully connected}) agregujące cechy wykryte przez poprzednie warstwy i służące do końcowej klasyfikacji.\cite{Goodfellow-et-al-2016}

Zaletą sieci konwolucyjnych jest automatyczne uczenie się reprezentacji cech z danych, co eliminuje konieczność ręcznego projektowania cech.

\begin{figure}
    \centering
    \includegraphics[width=0.6\linewidth]{konwolucja.png}
    \caption{Wizualizacja warstwy konwolucyjnej \cite{CNN-Explainer}}
    \label{fig:Wizualizacja warstwy konwolucyjnej}
\end{figure}


Jeśliby traktować rozpoznawanie zrzutów ekranu jako zwykłe zadanie klasyfikacji, to dowolna modyfikacja strony będzie skutkowała uzyskaniem niepoprawnych wyników. Wynika to z faktu, że większość treści w internecie jest dynamiczna, skierowana do konkretnego użytkownika. 

Dlatego, autorzy skupiają się na rozpoznawaniu logo firmy lub wyuczeniu modelu, tak by zbudować uniwersalny profil danej strony. 

Oba podejścia wymagają sporej elastyczności a jednym z powodów jest 
współistnienie wiele wariantów logo opisujące tą samą firmę, jak widać na rysunku poniżej:

\begin{figure}[hh]
    \centering
    \includegraphics[width=0.5\linewidth]{Rozne logo.png}
    \caption{Warianty logo firmy Adobe}
    \label{fig:rozne-logo}
\end{figure}

Wobec tego, autorzy\cite{Phishpedia} wykorzystują sieci syjamskie, które polegają na wytrenowaniu sieci na jednym zadaniu a następnie wzięciu wytrenowanego modelu i zastosowaniu go w innym zadaniu. Sieć jest trenowana jednocześnie na zbiorze Logo-2K+\cite{Logo2K} oraz logach ze zbioru danych, używając odległości kosinusowej by nauczyć się podobieństwa logo. W ten sposób model nie jest zmuszany do wyznaczenia uniwersalnej reprezentacji różnych logo tej samej firmy. Następnie porównywane są logo referencyjne z logo analizowanej strony i sprawdzane jest podobieństwo między nimi. Do tego celu jest wykorzystywany model Resnetv2 \cite{resnetv2}.

Celem jest ułożenie przestrzeni zanurzeń tak by przykłady pozytywne miały mniejszą odległość niż negatywne. Funkcja straty \textit{triplet loss} jest określana wzorem:
% Funkcję (zanurzenia) \( \psi \) można uzyskać ucząc model przy pomocy odpowiednio dobranej funkcji celu – w ogólności, odległość \( ||\psi(x_i) - \psi(x_j)|| \) (którą umiemy obliczyć, bo \( \tilde{X} \subseteq \mathbb{R}^m \)) :
% \begin{itemize}
%     \item \textbf{minimalizujemy}, gdy \( x_i, x_j \in X \) są bliskie/podobne/często występują wspólnie w sekwencjach,
%     \item \textbf{maksymalizujemy} – w przeciwnym przypadku.
% \end{itemize}
\begin{equation}
L(\mathbf{x}_a, \mathbf{x}_p, \mathbf{x}_n) = \max \big( 0, \, d\big(\psi(\mathbf{x}_a), \psi(\mathbf{x}_p)\big) 
- d\big(\psi(\mathbf{x}_a), \psi(\mathbf{x}_n)\big) + m \big),
\end{equation}
gdzie:
\begin{itemize}
    \item $\mathbf{x}_a$ -- tzw. kotwica (\textit{anchor}),
    \item $\mathbf{x}_p$ -- wektor pozytywny, który ma być "blisko" kotwicy
    \item $\mathbf{x}_n$ -- wektor negatywny, który ma być "daleko" kotwicy
    \item $m \in \mathbb{R}_+$ --  margines – zabezpieczenie przeciwko nakładaniu się punktów
    \item $d$ -- funkcja odległości.
\end{itemize}

Do rozpoznawania na zrzucie ekranu loga firm i pól tekstowych do wprowadzania danych autorzy wykorzystują model Faster-RCNN \cite{FasterR-CNN} modyfikując go tak aby wspólnie trenować sieć do rozpoznawania regionów i model Fast-RCNN.

Natomiast \cite{VisualPhishNet} opisuje wykorzystanie sieci syjamskich z trzema podsieciami neuronowymi, które na tej samej zasadzie próbują nauczyć się uniwersalnej reprezentacji stron internetowych. Na pierwszym etapie następuje losowe dobieranie przykładów. W drugim etapie, sieć jest uczona na 'trudnych' przykładach, tzn. tych które zostały niepoprawne sklasyfikowane. Podczas klasyfikacji, obliczana jest odległość między analizowanym zrzutem ekranu a każdą z monitorowanych stron. Cel ataku jest rozpoznawany jako strona o najmniejszej odległości.

\subsection{Zbiory danych}
Zadanie rozpoznawania celu ataku phishingu jest bardzo trudne. Uwaga jest skierowana na bardziej znane podmioty, opierając się na intuicji, atakujący dążą do maksymalizowania zysków z ataków, wobec czego skupiają się na podszywaniu się pod instytucje finansowe i znane marki\cite{PhishingArticle}
% (TODO: Tyler Moore. Phishing and the economics of e-crime.Infosecurity, 4(6):34–37, 2007) 
takie jak: firmy kurierskie, urzędy administracji, operatorów telekomunikacyjnych, czy nawet znajomych użytkownika, starają się wyłudzić dane do logowania np. do kont bankowych lub używanych przez atakowanego kont społecznościowych, czy systemów biznesowych.\cite{govPhishing}

Obecnie strony internetowe są często aktualizowane. Zmieniane są czcionki, układ strony i inne elementy wizualne. Ponadto zawierają dynamiczne treści, co utrudnia identyfikację. Kolejnym elementem jest fakt, że firmy mogą mieć logo w różnych wariantach oraz różne domeny np. "amazon.pl" oraz "amazon.co.jp" w zależności od kraju. To utrudnia pracę zwiększając liczbę wariantów stron należących do tej samego podmiotu, wymuszając większą liczbę stron, które należy analizować.

Jak zauważają autorzy pracy \cite{Phishpedia} liczba stron wcale może nie być taka duża. Gdyż w zbiorze około 30 000 stron phishingowych uzyskanych z systemu OpenPhish\footnote{\href{https://www.openphish.com/}{https://www.openphish.com/}}, 100 najpopularniejszych marek pokrywa aż 95.8\% przykładów. Potwierdzając intuicję o tym że atakujący wybierają jako celu ataku popularniejsze strony.


% % Równanie typu 'inline':
% \lipsum[2] $F = m \cdot a$ lorem ipsum dolor sit amet.
% % Równanie bez numeru
% % align oznacza wyrównanie kolejnych wierszy do '&'
% % '&' służy tylko do wyrównania i nie jest renderowany
% \begin{align*}
%     E & = mc^2 \\
%     y & = ax^2 + bx + c
% \end{align*}

% \lipsum[3]
% % Równanie numerowane: macierze
% \begin{align}
%     \begin{bmatrix}
%         1 & 0 & 0 \\
%         0 & 2 & 0 \\
%         0 & 0 & 3
%     \end{bmatrix} \cdot
%     \begin{bmatrix}
%         4 \\
%         5 \\
%         6
%     \end{bmatrix} =
%     \begin{bmatrix}
%         4  \\
%         10 \\
%         18
%     \end{bmatrix}
% \end{align}

% % Cytaty dla zapełnienia bibliografii
% \lipsum[4] Lorem ipsum dolor sit amet, consectetur adipiscing elit, sed do eiusmod tempor incididunt ut labore et dolore magna aliqua \cite{szczypiorski2015}, \cite{duqu2011}, \cite{shs2015}, \cite{wozniak2018}, \cite{dcp19}.

% % Podrozdział pierwszego poziomu
% \subsection{Critique of Pure Reason}
% \kant[1]

% % Tabela wielostronicowa, 4 kolumny
% % Kolumny typu m{} oznaczają kolumny o stałej szerokości z zawijaniem wierszy
% % Wyrównywane są domyślnie do lewej; aby ustawić inne wyrównanie,
% % stosujemy \multicolumn{1} tak jak poniżej
% \begin{longtable}{| c | m{0.58\linewidth} | r | m{0.1\linewidth} |}
%     \caption{Tabela wielostronicowa.}
%     \label{table:koszty} \\

%     \hline
%     % Nagłówek tabeli wyrównujemy do środka
%     Lp & \multicolumn{1}{c|}{Treść} & \multicolumn{1}{c|}{Kwota} & \multicolumn{1}{m{0.1\linewidth}|}{Wariant opłaty} \\ \hline\hline \endfirsthead \endfoot
%     \hline \endlastfoot

%     1  & Lorem ipsum dolor sit amet, consectetur adipiscing elit, sed do eiusmod tempor incididunt ut labore et dolore magna aliqua. & 111 111,11 zł & \multicolumn{1}{c|}{WAR1} \\ \hline
%     2  & Lorem ipsum dolor sit amet, consectetur adipiscing elit, sed do eiusmod tempor incididunt ut labore et dolore magna aliqua. & 22 222,22 zł & \multicolumn{1}{c|}{WAR1} \\ \hline
%     3  & Lorem ipsum dolor sit amet, consectetur adipiscing elit, sed do eiusmod tempor incididunt ut labore et dolore magna aliqua. & 33 333,33 zł & \multicolumn{1}{c|}{WAR1} \\ \hline
%     4  & Lorem ipsum dolor sit amet, consectetur adipiscing elit, sed do eiusmod tempor incididunt ut labore et dolore magna aliqua. & 444 444,44 zł & \multicolumn{1}{c|}{WAR1} \\ \hline
%     5  & Lorem ipsum dolor sit amet, consectetur adipiscing elit, sed do eiusmod tempor incididunt ut labore et dolore magna aliqua. & 55 555,55 zł & \multicolumn{1}{c|}{WAR1} \\ \hline
%     6  & Lorem ipsum dolor sit amet, consectetur adipiscing elit, sed do eiusmod tempor incididunt ut labore et dolore magna aliqua. & 66 666,66 zł & \multicolumn{1}{c|}{WAR1} \\ \hline
%     7  & Lorem ipsum dolor sit amet, consectetur adipiscing elit, sed do eiusmod tempor incididunt ut labore et dolore magna aliqua. & 777 777,77 zł & \multicolumn{1}{c|}{WAR1} \\ \hline
%     8  & Lorem ipsum dolor sit amet, consectetur adipiscing elit, sed do eiusmod tempor incididunt ut labore et dolore magna aliqua. & 8 888,88 zł & \multicolumn{1}{c|}{WAR1} \\ \hline
%     9  & Lorem ipsum dolor sit amet, consectetur adipiscing elit, sed do eiusmod tempor incididunt ut labore et dolore magna aliqua. & 999 999,99 zł & \multicolumn{1}{c|}{WAR1} \\ \hline
%     10 & Lorem ipsum dolor sit amet, consectetur adipiscing elit, sed do eiusmod tempor incididunt ut labore et dolore magna aliqua. & 111 111,11 zł & \multicolumn{1}{c|}{WAR2} \\ \hline
%     11 & Lorem ipsum dolor sit amet, consectetur adipiscing elit, sed do eiusmod tempor incididunt ut labore et dolore magna aliqua. & 22 222,22 zł & \multicolumn{1}{c|}{WAR2} \\ \hline
%     12 & Lorem ipsum dolor sit amet, consectetur adipiscing elit, sed do eiusmod tempor incididunt ut labore et dolore magna aliqua. & 33 333,33 zł & \multicolumn{1}{c|}{WAR2} \\ \hline
%     13 & Lorem ipsum dolor sit amet, consectetur adipiscing elit, sed do eiusmod tempor incididunt ut labore et dolore magna aliqua. & 444 444,44 zł & \multicolumn{1}{c|}{WAR2} \\ \hline
%     14 & Lorem ipsum dolor sit amet, consectetur adipiscing elit, sed do eiusmod tempor incididunt ut labore et dolore magna aliqua. & 55 555,55 zł & \multicolumn{1}{c|}{WAR2} \\ \hline
%     15 & Lorem ipsum dolor sit amet, consectetur adipiscing elit, sed do eiusmod tempor incididunt ut labore et dolore magna aliqua. & 66 666,66 zł & \multicolumn{1}{c|}{WAR2} \\ \hline
%        & \multicolumn{1}{r|}{\textbf{Suma:}} & \textbf{7 777 777,77 zł} &
% \end{longtable}

% \kant[2]

% % Nagłówki kolejnych poziomów, dla zapełnienia spisu treści
% \subsection{Caegorical Imperative} % 2.2
% \subsubsection{Deontological Ethics} % 2.2.1
% \kant[2]
% \subsubsection{Consequentialism -- the Ideal of practical reason} % 2.2.2
% \kant[3]
% \subsection{G\"odel's ontological proof} % 2.3
% \kant[9] Lorem ipsum dolor sit amet, consectetur adipiscing elit \cite{benzmuller2014}, \cite{goedel95}, \cite{wang97}, \cite{koons2005}.

% % Twierdzenia i dowody
% % Założenie
% \begin{assumption} \label{ass:1}
%     $ [\![ \ \phi \ ]\!] \Longrightarrow [\![ \ P(\phi); \neg P(\phi) \ ]\!]$
% \end{assumption}
% % Aksjomat
% \begin{axiom}[Dualność] \label{axiom:1}
%     $\neg P(\phi) \Leftrightarrow P(\neg \phi)$, równoważnie $P(\phi) \Leftrightarrow \neg P(\neg \phi)$
% \end{axiom}
% \begin{axiom}[Całkowitość] \label{axiom:2}
%     $ \left( P(\phi) \wedge \forall x: \phi(x) \Rightarrow \psi(x) \right) \Rightarrow P(\psi) $
% \end{axiom}
% \begin{axiom}[Absolutność] \label{axiom:3}
%     $ P(\phi) \Rightarrow \Box P(\phi) $
% \end{axiom}
% % Definicja
% \begin{definition} \label{def:1}
%     $ G(x) \Leftrightarrow \forall \phi: \left( P(\phi) \Rightarrow \phi(x) \right) $
% \end{definition}
% \begin{definition} \label{def:2}
%     $ \phi \ ess \ x \Leftrightarrow \phi(x) \wedge \forall \psi \left( \psi(x) \Rightarrow \Box \forall y \left( \phi(y) \Rightarrow \psi(y) \right) \right)  $
% \end{definition}
% \begin{axiom} \label{axiom:4}
%     P(G)
% \end{axiom}
% % Lemat
% \begin{lemma} \label{lemma:1}
%     $ P(\phi) \Rightarrow \Diamond \exists x : \phi(x) $
% \end{lemma}
% \begin{proof}
%     Dowód pomijamy, bo jest trywialny :)
% \end{proof}
% \begin{lemma} \label{lemma:2}
%     $ \Diamond \exists x : G(x) $
% \end{lemma}
% \begin{proof}
%     Natychmiastowy wniosek z aksjomatu \ref{axiom:4} i lematu \ref{lemma:1}.
% \end{proof}
% \begin{lemma} \label{lemma:3}
%     $ G(x) \Rightarrow G \ ess \ x $
% \end{lemma}
% \begin{proof}
%     Poprzez podstawienie do definicji \ref{def:2}.
% \end{proof}
% \begin{definition} \label{def:3}
%     $ E(x) \Leftrightarrow \forall \phi \left( \phi \ ess \ x \Rightarrow \Box\ \exists x: \phi(x) \right) $
% \end{definition}
% \begin{axiom} \label{axiom:5}
%     P(E)
% \end{axiom}
% % Twierdzenie
% \begin{theorem}
%     $ \Box\ \exists x : G(x) $
% \end{theorem}
% \begin{proof}
%     Na podstawie definicji \ref{def:1}, lematu \ref{lemma:3} i aksjomatu \ref{axiom:5}.
% \end{proof}
