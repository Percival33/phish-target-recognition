\clearpage % Rozdziały zaczynamy od nowej strony.
\section{Wstęp}
\subsection{Wprowadzenie}
Wraz z postępęm technologicznym równie szybko postępuje rozwój internetu co przekłada się na rozwój cyfryzacji. Skutkuje to coraz większą liczbą stron internetowych oraz faktem że coraz więcej firm oferuje swoje usługi online. Dodatkowo powstają systemy, które ułatwiają dostęp do danych. 
Takimi systemami są między innymi systemy bankowe czy też rządowe jak Internetowe Konto Pacjenta (IKP) \footnote{https://pacjent.gov.pl/}. 
W obliczu takowego rozwoju powstaje również coraz więcej oszustw internetowych. 
Począwszy od fałszywych sklepów internetowych, złośliwego oprogramowania przez podejrzane załączniki w e-mailach aż do złośliwych domen wyłudzających dane osobowe lub środki finansowe. 
Te ostatnie nazywane są również phishingiem. Zapobieganie tej formie cyberprzestępczości jest jednym z zadań zespołu CERT Polska \cite{certpl}, z którym nawiązałem współpracę podczas realizacji tej pracy. 

\subsection{Cel pracy}
Celem tej pracy jest stworzenie systemu rozpoznawania stron phishingowych na podstawie podobieństwa wizualnego oraz rozpoznawanie celu ich podszycia.
Ma on być zbudowany w celu ułatwienia pracy zespołu CERT Polska w wykrywaniu stron phishingowych, informowaniem o potencjalnych zagrożeniach oraz udaremnianiem ataków.
System ten ma pozwalać na łatwe dodawanie nowych metod rozpoznawania oraz nowych danych wejściowych.
Kolejnym celem jest wykorzystanie uczenia maszynowego, które na podstawie wyników z różnych metod, wyda ostateczną decyzję o prawdziwości strony i w przypadku fałszywej strony, rozpozna cel ataku.
Działania tej organizacji zainspirowały mnie do zgłębienia tematyki phishingu oraz metod ich zapobiegania jako że jest to temat, który dotyka ludzi na co dzień coraz częściej.
Dodatkowo CERT Polska dostarczył zbiór danych, który posłużył do ewaluacji systemu, stworzonego w tej pracy.

Przy okazji budowania systemu, zostanie przybliżona tematyka phishingu.
Zostaną zbadane i ewaluowane istniejące rozwiązania rozpoznawania na podstawie podobieństwa wizualnego

\subsection{Struktura pracy}
TODO: poprawić ten tekst, czas przyszły czy przeszły?
W rozdziale drugim zostanie przedstawiona wiedza teoretyczna, na której opierają się zastosowane rozwiązania (?) z przeglądu literatury naukowej w tym zakresie. Przegląd ten został on przeprowadzony w rozdziale trzecim. Dalej w rozdziale czwartym najpierw wymienione są wymagania wobec systemu i decyzje projektowe, które zostały podjęte aby takowe wymagania spełnić. Kolejny rozdział traktuje o implementacji systemu, w którym opisane są technologie w zrealizowanym systemie a także opis w jaki sposób zostały one zintegrowane ze sobą. W rozdziale szóstym opisane są eksperymenty, które zostały przeprowadzone w celu sprawdzenia skuteczności systemu. Zostały tam porównane każda z metod na każdym zbiorze danych z literatury naukowej oraz na zbiorze przygotowanym przez CERT. Na końcu znajduje się podsumowanie pracy, w którym zawarte są wnioski z przeprowadzonych eksperymentów oraz możliwości dalszego rozwoju systemu.

% % Akapit z cytatem
% \lipsum[1] \cite{goossens93}

% % Przykładowy obrazek
% \begin{figure}[!h]
%     % Wyrównanie obrazka, szerokość i plik
%     % Zamiast width można też użyć height, etc.
%     \centering \includegraphics[width=0.5\linewidth]{logopw.png}
%     % Podpis umieszczamy pod obrazkiem
%     % znacznik \caption służy również do wygenerowania numeru obrazka
%     \caption{Tradycyjne godło Politechniki Warszawskiej}
%     % \label pozwala odwołać się do obrazka w innych miejscach za pomocą \ref
%     % odwołanie \ref renderuje się jako numer obrazka,
%     % dlatego zawsze najpierw używaj \caption a potem \label
%     \label{fig:tradycyjne-logo-pw}
% \end{figure}

% % \ref wyrenderuje się jako 'Reference to image 1.1'
% \lipsum[2] Reference to image \ref{fig:tradycyjne-logo-pw}.

% % Lista punktowana
% % Parametr label ustawia symbol punktora
% \begin{itemize}
%     \item Item 1:
%     \begin{itemize}[label=---]
%         \item item 1.1;
%         \item item 1.2;
%     \end{itemize}
%     \item Item 2;
%     \item Item 3.
% \end{itemize}

% \lipsum[3]

% % Lista numerowana w formacie 1.a).ii
% % Tutaj również można stosować \label
% \begin{enumerate}
%     \item Item 1:
%     \begin{enumerate}
%         \item item 1.1;
%         \item item 1.2:
%         \begin{enumerate}
%             \item item 1.2.1;
%             \item item 1.2.2;
%         \end{enumerate}
%         \item item 1.3;
%     \end{enumerate}
%     \item Item 2;
%     \item Item 3.
% \end{enumerate}

% % Przypis dolny \footnote
% \lipsum[4] Lorem ipsum dolor sit amet\footnote{Lorem ipsum dolor sit amet, consectetur adipiscing elit, sed do eiusmod tempor incididunt ut labore et dolore magna aliqua. Ut enim ad minim veniam, quis nostrud exercitation ullamco laboris nisi ut aliquip ex ea commodo consequat.}, consectetur adipiscing elit.

% % Przykładowa tabela: wyśrodkowana i renderowana
% % w miejscu wstawienia: !h = !h[ere]
% % Domyślnie tabele trafiają na górę strony
% \begin{table}[!h] \centering
%     % Podpis tabeli umieszczamy od góry
%     \caption{Przykładowa tabela.}
%     \label{tab:tabela1}

%     % Tabela z trzema kolumnami:
%     % dwie wyrównanie do środka [c], a ostatnia do prawej [r]
%     % szerokość kolumn automatyczna (równa szerokości tekstu)
%     \begin{tabular}{| c | c | r |} \hline
%         Kolumna 1       & Kolumna 2 & Liczba \\ \hline\hline
%         cell1           & cell2     & 60     \\ \hline
%         cell4           & cell5     & 43     \\ \hline
%         cell7           & cell8     & 20,45  \\ \hline
%         % Komórka o szerokości dwóch kolumn, wyrównana do prawej
%         % Przypisy dolne w tabelach wstawiamy przez \tablefootnote
%         \multicolumn{2}{|r|}{Suma\tablefootnote{Table footnote.}} & 123,45 \\ \hline
%     \end{tabular}

% \end{table}

% Lorem ipsum dolor sit amet.
